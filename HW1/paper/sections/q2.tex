\section*{Exercise 2}
\prob{1}{
Assuming that the variables $z$ in the dataset is a valid instrument for prices, write down the moment condition that allows you to consistently estimate $(\alpha, \beta)$ and obtain an estimate for both parameters.
}

../../tasks/q2/output/iv_results.tex

Taking the ratio of product $j$ 's share to the outside good's share and taking logs yields:

\begin{align*}
\ln \left(\frac{s_{j t}}{s_{0 t}}\right)=-\alpha p_{j t}+\beta x_{j t}+\xi_{j t} .
\end{align*}


Denote

\begin{align*}
y_{j t}=\ln \left(s_{j t}\right)-\ln \left(s_{0 t}\right),
\end{align*}

so the linear model in the data is

\begin{align*}
y_{j t}=-\alpha p_{j t}+\beta x_{j t}+\xi_{j t} .
\end{align*}


Because $p_{j t}$ may be endogenous (correlated with $\left.\xi_{j t}\right)$, we use the instrument $z_{j t}$. The standard logit IV/ GMM moment condition is that the instrument is uncorrelated with the unobserved characteristic:

\begin{align*}
\mathbb{E}\bigl[z_{jt} \,\xi_{jt}\bigr] \;=\; 0.
\end{align*}

Equivalently, in sample,

\begin{align*}
\frac{1}{NT}\sum_{j,t} z_{jt} \,
\Bigl(y_{jt} + \alpha\,p_{jt} - \beta\,x_{jt}\Bigr)
\;=\; 0.
\end{align*}

\prob{2}{
2. For each market, compute own and cross-product elasticities. Average your results across markets and present them in a $J \times J$ table whose ( $i, j$ ) element contains the (average) elasticity of product $i$ with respect to an increase in the price of product $j$. What do you notice?
}

% latex table generated in R 4.4.2 by xtable 1.8-4 package
% Thu Jan 30 10:38:32 2025
\begin{table}[ht]
\centering
\begin{tabular}{rrrrrrr}
  \hline
 & Product1 & Product2 & Product3 & Product4 & Product5 & Product6 \\ 
  \hline
Product1 & -0.64 & 0.17 & 0.07 & 0.07 & 0.06 & 0.07 \\ 
  Product2 & 0.16 & -0.64 & 0.07 & 0.07 & 0.06 & 0.07 \\ 
  Product3 & 0.16 & 0.17 & -0.66 & 0.07 & 0.06 & 0.07 \\ 
  Product4 & 0.16 & 0.17 & 0.07 & -0.66 & 0.06 & 0.07 \\ 
  Product5 & 0.16 & 0.17 & 0.07 & 0.07 & -0.66 & 0.07 \\ 
  Product6 & 0.16 & 0.17 & 0.07 & 0.07 & 0.06 & -0.66 \\ 
   \hline
\end{tabular}
\caption{Average Own- and Cross-Price Elasticities (6x6)} 
\label{tab:elasticities}
\end{table}


In the homogeneous logit model:

\begin{align*}
s_j(\mathbf{p})=\frac{e^{\delta_j}}{1+\sum_{k=1}^J e^{\delta_k}}, \quad \delta_j=-\alpha p_j+\beta x_j+\xi_j
\end{align*}


The derivative of $s_j$ w.r.t. $p_k$ is:

\begin{align*}
\frac{\partial s_j}{\partial p_k}= \begin{cases}\alpha s_j\left(1-s_j\right), & k=j, \\ {\alpha s_j s_k,} & k \neq j\end{cases}
\end{align*}


Hence the price elasticity of $s_j$ with respect to $p_k$ is:

\begin{align*}
\varepsilon_{j, k}=\frac{\partial s_j}{\partial p_k} \frac{p_k}{s_j}= \begin{cases}\alpha p_j\left(1-s_j\right), & k=j, \\ { \alpha p_k s_k,} & k \neq j .\end{cases}
\end{align*}


\textbf{Implementation:}
\begin{enumerate}
\item For each market $t$ and each product $j$, plug in estimates $\hat{\alpha}$ and the observed $\left(p_{j t}, s_{j t}\right)$.
\item Compute the $6 \times 6$ elasticity matrix (own- and cross-) for each market.
\item Average across markets.
\end{enumerate}

We notice the IIA property, where cross-price elasticities are the same for all products (within product). This is a feature of the logit model, and is not always realistic.



\prob{3}{
Using your demand estimates, for each product in each market recover the marginal cost $c_{j t}$ implied by Nash-Bertrand competition. For simplicity, you can assume that in each market each product is produced by a different firm (i.e., there is no multi-products firms). Report the average (across markets) marginal cost for each product. Could differences in marginal costs explain the differences in the average (across markets) market shares and prices that you observe in the data?
}

../../tasks/q2/output/prices_and_shares.tex

We assume each product $j$ is produced by a single-product firm. The standard first-order condition for a firm producing only product $j$ in a logit demand model is:

\begin{align*}
\pi_j=\left(p_j-c_j\right) Q_j, \quad Q_j=M s_j .
\end{align*}


Taking derivative of $\pi_j$ w.r.t. $p_j$ and setting to 0 yields:

\begin{align*}
s_j+\left(p_j-c_j\right) \frac{\partial s_j}{\partial p_j}=0 \Longrightarrow p_j-c_j=-\frac{s_j}{\frac{\partial s_j}{\partial p_j}}=\frac{1}{\alpha\left[1-s_j\right]}
\end{align*}


Hence

\begin{align*}
c_j=p_j-\frac{1}{\alpha\left[1-s_j\right]}
\end{align*}


We use the estimated $\hat{\alpha}$ (and the observed $p_{j t}, s_{j t}$ ) to back out $c_{j t}$.


\prob{4}{
Suppose that product $j=1$ exits the market. Assuming that marginal costs and product characteristics for the other products remain unchanged, use your estimated marginal costs and demand parameters to simulate counterfactual prices and market shares in each market. Report the resulting average prices and shares.
}

% latex table generated in R 4.4.2 by xtable 1.8-4 package
% Thu Jan 30 10:38:34 2025
\begin{table}[ht]
\centering
\begin{tabular}{rrr}
  \hline
product & price\_cf\_avg & share\_cf\_avg \\ 
  \hline
  2 & 3.66 & 0.24 \\ 
    3 & 3.15 & 0.11 \\ 
    4 & 3.16 & 0.11 \\ 
    5 & 3.15 & 0.11 \\ 
    6 & 3.16 & 0.11 \\ 
   \hline
\end{tabular}
\caption{Counterfactual Average Prices and Shares (products 2-6)} 
\label{tab:cf}
\end{table}


Now suppose product 1 is removed. Everything else (the product characteristics ${ }^{x_j}$ the marginal cost $c_j$ ) stays the same. The other 5 products (and the outside good) re-solve the Nash-Bertrand equilibrium.

New Equilibrium Conditions

For single-product firms with the remaining $j=2, \ldots, 6$, each firm's FOC is

\begin{align*}
p_j-c_j=\frac{1}{\alpha\left[1-s_j(\mathbf{p})\right]}, \quad j=2, \ldots, 6 .
\end{align*}


But now the share $s_j(\mathbf{p})$ is computed from a 5 -product logit formula:

\begin{align*}
s_j(\mathbf{p})=\frac{\exp \left(\delta_j\left(p_j\right)\right)}{1+\sum_{k \in\{2, \ldots, 6\}} \exp \left(\delta_k\left(p_k\right)\right)}, \quad \delta_j=-\alpha p_j+\beta x_j+\xi_j
\end{align*}


This is a system of 5 equations in 5 unknowns $\left\{p_2, \ldots, p_6\right\}$, which we solve with fixed-point iteration.

\begin{enumerate}
    \item Initialize $\left\{p_2^{(0)}, \ldots, p_6^{(0)}\right\}$ at the old prices or at marginal cost + guess.
    \item Compute new shares $s_j^{(k)}$ using the logit formula with the current guess $\left\{p_2^{(k)}, \ldots, p_6^{(k)}\right\}$.
    \item Update each price using the best-response formula.
    \item  Iterate until convergence (i.e., $\max _j\left|p_j^{(k+1)}-p_j^{(k)}\right|<$ tolerance).
\end{enumerate}

\prob{5}{
Finally, for each market compute the change in firms' profits and in consumer welfare following the exit of firm $j=1$. Report the average changes across markets. Who wins and who loses?
}

% latex table generated in R 4.4.2 by xtable 1.8-4 package
% Thu Jan 30 10:38:34 2025
\begin{table}[ht]
\centering
\begin{tabular}{rrrr}
  \hline
product & avg\_profit\_base & avg\_profit\_cf & avg\_delta\_profit \\ 
  \hline
  1 & 1.08 & 0.00 & -1.08 \\ 
    2 & 1.09 & 1.38 & 0.29 \\ 
    3 & 0.42 & 0.54 & 0.12 \\ 
    4 & 0.41 & 0.53 & 0.12 \\ 
    5 & 0.41 & 0.53 & 0.12 \\ 
    6 & 0.42 & 0.54 & 0.12 \\ 
   \hline
\end{tabular}
\caption{Average Baseline vs. Counterfactual Profits by Product} 
\label{tab:profit}
\end{table}


../../tasks/q2/output/consumer_surplus.tex

Profits: $\pi_j=\left(p_j-c_j\right) Q_{j .}$ Product 1 's profit goes to zero. The remaining firms' prices, profits, and market shares increase (with product 2 firm gaining the most as the closest substitute).

Consumer Surplus: $\Delta \text{CS}
\;=\;
\frac{1}{\alpha}\Bigl[\ln\bigl(1 + \sum_{j\in CF} e^{\delta_j^1}\bigr)
\;-\;
\ln\bigl(1 + \sum_{j\in Base} e^{\delta_j^0}\bigr)\Bigr].
$

Consumer surplus falls from higher prices and less variety. 
