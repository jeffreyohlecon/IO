\documentclass[12pt]{article}
\usepackage{amsmath,amssymb,graphicx}
\usepackage[margin=1in]{geometry}
\usepackage{booktabs} % for nice tables

\title{PS2}
\date{\today}

\begin{document}
\maketitle

\section{Exercise 1}
Our model has two actions, \emph{maintain} (action 1) or \emph{replace} (action 2), and the state $x_t \in \{0,\ldots,X-1\}$ denotes mileage. The deterministic component of the flow utilities is parameterized by
\[
\overline{u}_1(x_t;\theta) = \theta_1 \, x_t, 
\quad
\overline{u}_2(x_t;\theta) = \theta_2,
\]
and the agent’s discount factor is $\beta$. The mileage evolves according to a Markov chain with transition probabilities $\varphi_1,\varphi_2$ for increments in mileage. The unobserved shocks follow a T1EV(0,1) distribution.

\section{Question 1: Varying Parameters and CCP Plots}
\label{sec:Q1}
\subsection*{(a) Plotting the CCPs}
We first generate plots of the implied true conditional choice probabilities (CCPs) of \emph{maintaining the engine} (action 1) for each mileage state $x_t$ under different parameter values of $(\beta,\theta_1,\theta_2)$. 

Figure~\ref{fig:param-ccp} shows the probability $\Pr(\text{maintain}\mid x)$ for several parameter combinations. As seen in the figure:
\begin{itemize}
\item \textbf{Varying $\beta$:} As $\beta$ increases (i.e., the agent discounts the future less), the probability of maintenance decreases at all mileage levels. 
\item \textbf{Varying $\theta_1$:} If $\theta_1$ is more negative, so maintenance costs are higher, the probability of maintenance decreases at all mileage levels.
\item \textbf{Varying $\theta_2$:} Higher $\theta_2$ (higher replacement cost) raises the probability of maintenance at all mileage levels. This is obvious: more costly replacement makes maintenance more attractive.
\end{itemize}
Overall, the patterns match economic intuition and confirm that the CCPs respond sensibly to parameter changes.

\begin{figure}[h!]
    \centering
    \includegraphics[width=0.55\textwidth]{param_ccp_example.png}
    \caption{Probability of maintaining vs.\ mileage, under different $(\beta,\theta_1,\theta_2)$.}
    \label{fig:param-ccp}
\end{figure}

\section{Question 2: Hotz--Miller CCP Inversion Implementation}
We next implement a simplified version of Hotz and Miller (1993)’s choice-probability inversion theorem. The key idea is that under T1EV errors, the ratio of the CCPs identifies the difference in choice-specific value functions:
\[
\ln\!\Bigl(\frac{\Pr(a=j\mid x)}{\Pr(a=J\mid x)}\Bigr)
\;=\;
\overline{v}_j(x) - \overline{v}_J(x).
\]
With a reference choice $J$ normalized to zero, we recover $\overline{v}_j(x)$ from the sample analog $\widehat{\Pr}(a=j\mid x)$. The code calculates these inverted value functions and then sets up the (partial) likelihood for the structural parameters $\theta_1$ and $\theta_2$.

\section{Question 3: Estimation with the CCP Inversion}
\label{sec:Q3}
\subsection*{(a) Simulation and Parameter Recovery}
Using our \texttt{draw} function, we simulate data under ``true'' parameters (e.g.\ $\theta_1=-0.25,\;\theta_2=5,\;\beta=0.95$). We estimate $\theta=(\theta_1,\theta_2)$ by maximizing the two-step Hotz--Miller likelihood. Figure~\ref{fig:twostep-ccp} compares the \emph{true} CCP and the \emph{estimated} CCP from the two-step procedure. Numerically, the estimated parameters are close to the true values, and the estimated CCP curve aligns well with the true CCP curve across the mileage states.

\begin{figure}[h!]
    \centering
    \includegraphics[width=0.55\textwidth]{twostep_ccp_example.png}
    \caption{True CCP vs.\ estimated CCP (two-step). Parameter estimate: $(\theta_1,\theta_2)\approx(-0.258,\,4.998)$.}
    \label{fig:twostep-ccp}
\end{figure}

\section{Question 4: Comparing NFXP, NPL, and MPEC}
We repeat the estimation on the same simulated data using:
\begin{enumerate}
\item \textbf{Nested Fixed-Point (NFXP):} 
   The classical approach that repeatedly solves the Bellman equation in an inner loop while searching over $(\theta_1,\theta_2)$ in the outer loop.
\item \textbf{Nested Pseudo-Likelihood (NPL):} 
   An iterative procedure that updates CCPs and parameters in turn, often faster than full NFXP if well-implemented.
\item \textbf{MPEC (Mathematical Programming with Equilibrium Constraints):}
   A constrained optimization approach that imposes the Bellman fixed-point as constraints and maximizes the likelihood subject to those constraints.
\end{enumerate}

We time each method. Below is the output (with parameter estimates and runtimes):

\begin{verbatim}
True theta: [-0.25  5.  ]
Beta: 0.95
NFXP estimated theta: [-0.27103871  5.8225678 ]
NFXP runtime: 8.1379 seconds

NPL estimated theta: [-0.29438793  5.38765412]
NPL runtime: 94.7935 seconds

MPEC estimated theta: [-0.34869353  5.43107649]
MPEC runtime: 119.3358 seconds
\end{verbatim}

\textcolor{red}{DO FOR OTHER VALUES}


\noindent\textbf{Results}
\begin{itemize}
\item All three methods yield similar estimates, which are close to the true values (I am a bit surprised by how far off they are though).
\item The differences in run times is stark: NFXP is the fastest, followed by NPL, and MPEC is the slowest. The jump from NFXP to NPL is very consequential.
\item In practice, I guess balance the computational cost with implementation complexity; NPL and the two-step CCP approach can be more scalable in certain applications, while NFXP remains a benchmark method, and MPEC can be elegant with powerful solvers.
\end{itemize}

\end{document}
